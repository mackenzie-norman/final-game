\documentclass{article}
\usepackage{listings}
\usepackage{xcolor}
\usepackage{graphicx}
\usepackage{float}
\definecolor{codegreen}{rgb}{0,0.6,0}
\definecolor{codegray}{rgb}{0.5,0.5,0.5}
\definecolor{codepurple}{rgb}{0.58,0,0.82}
\definecolor{backcolour}{rgb}{0.95,0.95,0.92}

\lstdefinestyle{mystyle}{
    backgroundcolor=\color{backcolour},   
    commentstyle=\color{codegreen},
    keywordstyle=\color{magenta},
    numberstyle=\tiny\color{codegray},
    stringstyle=\color{codepurple},
    basicstyle=\ttfamily\footnotesize,
    breakatwhitespace=false,         
    breaklines=true,                 
    captionpos=b,                    
    keepspaces=true,                 
    numbers=left,                    
    numbersep=5pt,                  
    showspaces=false,                
    showstringspaces=false,
    showtabs=false,                  
    tabsize=2
}

\lstset{style=mystyle}

\title{The life and times of Michael K.}
\author{Mackenzie Norman}
\begin{document}
\maketitle
\begin{center}
    \includegraphics[width=\textwidth]{title_screen.png}
\end{center}
\section*{Instructions}
The goal of the game (if you could call it that) is to garden. You begin the game with 10 pumpkin seeds and 2 Melon seeds.

Which can be planted by first \textbf{selecting them from the menu on the left} and then \textbf{clicking anywhere on the screen}

\textbf{Note:} You can always tell which menu option is selected because it will be highlighted in green.

    \begin{figure}[H]
    \includegraphics[width=\textwidth]{melons-pumpkins.png}
    \caption{Menu Option for Pumpkins and Melons}
        
    \end{figure}

Melons and Pumpkins cannot be planted on top of eachother, or on \textbf{weeds} which will grow randomly as the game progresses

The next thing you must do in order for your plants to grow is water them. You can do so by \textbf{selecting the water option from the menu on the left} which will allow you to click and water your plants. You can tell how well watered a plant is by the dark spot surrounding it. 

You water will run out as you water, to refill it, simply click the \textbf{pond} in the upper right corner, which will refill the water level.

    \begin{figure}[H]
    \includegraphics[width=\textwidth]{watering.png }
    \caption{Watering Instructions}
        
    \end{figure}

The first menu option is the hoe, this allows you to harvest Pumpkins and Melons for seeds by \textbf{clicking them}, and to remove weeds(if you want to, weeds aren't really harmful). Pumpkins and Melons can only be harvested when they are fully grown. It will take some time to learn what this looks like so be patient.

The final menu option is \textbf{Bird Mode} clicking this will take to a view where you are overlooking your garden from the perspective of a bird. While in this mode you cannot interact with your crops, but they will continue to grow (as will weeds).

To leave this mode, \textbf{click the large green arrow in the upper left}
    \begin{figure}[H]
    \includegraphics[width=\textwidth]{Screenshot 2025-06-05 105052.png}
    \caption{Bird Mode, note the green arrow in the left}
        
    \end{figure}






\section*{Why}
This game came to me as I was writing my keyword project. I think the life and times of Michael K. is one of the more transformative books I have had the privilege of reading. I felt the same way as when I read The Vegetarian (which is very high praise).
K. wants solely to be a gardener so I thought it would be interesting to make a game in which you can do that. I think Michael's hunger is one of the more interesting part of the story and so I struggled to add it to the game. Initially I added a variable \lstinline[language=c]|hunger| so that the game could track the hunger of Michael, and this is how you would lose (death being loss). But as I went on, I thought about how Michael did not seem to get hungry when he was gardening ``Am I to believe that you lived for a year on pumpkin? The human body is not capable of that, Michaels.'' 
I think this is Coetzee's `response' to idea of K. being this ``ethical indian''. By not succumbing to hunger while gardening K. takes the role of a machine, akin to a combine, no longer fully human, simply waiting for his time to harvest. 
I wanted my game to be peaceful and possible to played idly (or even by doing nothing), which is why I added weeds. If you sit long enough the farm will be completely taken over by weeds. The choice as the player is how you want to labor; should you water your crops all the time, you can easily spend all your time watering (this is why I switched the water to being something you fill up), or all your time fighting weeds, or you can plant your seeds and sit passively, waiting for what you planted to be ready to harvest.

Additionally the game supports the idea of the land being something that you have control over, ``The first thing is the land'' So with this game, you are able to control the land as you see fit. 

I also included the birds eye view to support Michaels dream of ``I used to think about flying. I always wanted to fly. I used to stretch out my arms and think I was flying over the fences and between the houses. I flew low over people's heads, but they couldn't see me.'' 



\end{document}